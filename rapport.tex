\documentclass[12pt,a4paper]{article}

\usepackage[utf8]{inputenc}
\usepackage[T1]{fontenc}
\usepackage[swedish]{babel}
\usepackage{amsmath}
\usepackage{ae}
\usepackage{units}
\usepackage{icomma}
\usepackage{color}
\usepackage{graphicx}
\usepackage{bbm}
\usepackage{wrapfig}
%\usepackage{epsfig}
\usepackage[retainorgcmds]{IEEEtrantools}
\usepackage{hyperref}
\hypersetup{colorlinks,
	citecolor=black,
	filecolor=black,
	linkcolor=black,
	urlcolor=black,
	pdftex}
\newcommand{\N}{\ensuremath{\mathbbm{N}}}
\newcommand{\Z}{\ensuremath{\mathbbm{Z}}}
\newcommand{\Q}{\ensuremath{\mathbbm{Q}}}
\newcommand{\R}{\ensuremath{\mathbbm{R}}}
\newcommand{\C}{\ensuremath{\mathbbm{C}}}
\newcommand{\rd}{\ensuremath{\mathrm{d}}}
\newcommand{\id}{\ensuremath{\,\rd}}
\newcommand{\degree}{\ensuremath{^{\circ}}}


\begin{document}
	\pagenumbering{roman}

\title{Analytisk och nummerisk undersökning av modell för oscillerande partiklarupphängda i ideala fjädrar i en spatiell dimension}
	\author{AUTHORS}
	\date{DATE}
	\maketitle
	\thispagestyle{empty}

	\begin{abstract}
		ABSTRACT
	\end{abstract}

\newpage

	\tableofcontents
	\thispagestyle{empty}

\newpage

	\setcounter{page}{1}
	\pagestyle{plain}
	\pagenumbering{arabic}
\section{Uppgift 1, N=3}
Ur friläggning av de enskilda partiklarna finner vi att:

\begin{equation*}
m\ddot{q_1} = k(q_2-q_1)-kq_1 \\
m\ddot{q_2} = k(q_3-q_2)-k(q_2-q_1) \\
m\ddot{q_3} = k(q_3-q_2)-kq_2
\end{equation*}

Vilket är ekvivalent med:

\begin{equation*}
\ddot{q_1} + \frac{2k}{m}q_1 - \frac{k}{m}q_2 = 0 \\
\ddot{q_2} + \frac{2k}{m}q_2-\frac{2}{m}q_1 - \frac{k}{m}q_3 = 0 \\
\ddot{q_3} + \frac{2k}{m}q_3 - \frac{k}{m}q_2=0
\end{equation*}

Vi inför beteckningarna: 

\begin{equation*}
\omega_0=\sqrt{k/m}
K=
\begin{bmatrix}
2 &-1 &0 \\
 -1 & 2 &-1\\
 0 &-1 &2
\end{bmatrix}\\
Q=
\begin{bmatrix}
q_1 \\ 
q_2 \\
q_3
\end{bmatrix}
\end{equation*}

Vårt system av differentialekvationer kan nu skrivas som:

\begin{equation*}
\ddot{Q} = \omega_0^2 KQ
\end{equation*}

Vi gör ansatzen:

\begin{IEEEeqnarray*}{rCl}
Q & = & A e^{i \omega t} \\
A & = &
\begin{bmatrix}
a_1 \\
a_2 \\
a_3
\end{bmatrix}
\end{IEEEeqnarray*}

Där $a_i, i=1,2,3$ är amplituder för respektive partikel. Vilket ger oss:

\begin{equation*}
\ddot{Q}=-\omega^2 Q = -\omega^2 A e^{i \omega t} \\
\end{equation*}

Och insatt i ekvationssystemet får vi:

\begin{equation}
e^{i \omega t}(-\omega^2 A + \omega_0^2 K A) =0 \\
<=> KA=\frac{\omega^2}{\omega_0^2} A
\end{equation}

Detta känns igen som ett egenvärdesproblem.
\end{document}
